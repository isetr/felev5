\documentclass[12pt]{article}

%margó méretek
\usepackage[a4paper,
inner = 25mm,
outer = 25mm,
top = 25mm,
bottom = 25mm]{geometry}


%packagek, ha használni akarunk valamit menet közben
\usepackage{lmodern}
\usepackage[magyar]{babel}
\usepackage[utf8]{inputenc}
\usepackage[T1]{fontenc}
\usepackage[hidelinks]{hyperref}
\usepackage{graphicx}
\usepackage{amssymb}
\usepackage{epstopdf}
\usepackage{setspace}
\usepackage[nottoc,numbib]{tocbibind}
\usepackage{setspace}

\setstretch{1.2}
\begin{document}
	
	

% a címlap, túl sokat nem kell módosítani rajta, leszámítva a neved/neptunodat (dátumot)
\begin{titlepage}
	\vspace*{0cm}
	\centering
	\begin{tabular}{cp{1cm}c}
		\begin{minipage}{4cm}
			\vspace{0pt}
			\includegraphics[width=1\textwidth]{elte_cimer}
		\end{minipage} & &
		\begin{minipage}{7cm}
			\vspace{0pt}Eötvös Loránd Tudományegyetem \vspace{10pt} \newline
			Informatikai Kar \vspace{10pt} \newline
			Programozási Nyelvek és Fordítóprogramok Tanszék
		\end{minipage}
	\end{tabular}
	
	\vspace*{0.2cm}
	\rule{\textwidth}{1pt}
	
	\vspace*{3cm}
	{\Huge Osztott rendszerek specifikációja és implementációja }
	
	\vspace*{0.5cm}
	{\normalsize IP-08bORSIG}
	
	\vspace{2cm}
	{\huge Dokumentáció az \textit{2}. beadandóhoz}
	
	\vspace*{5cm}
	
	{\large \verb|Szalóki Sándor| } % (név)
	
	{\large \verb|H8L59S| }  % (neptun)
		
	
	\vfill
	
	\vspace*{1cm}
	2017. december 1. % (dátum)
\end{titlepage}

\section{Kitűzött feladat}

Feladatunk annak eldöntése, hogy egy adott halmaznak létezik e olyan részhalmaza, melyben található elemek összege pontosan megegyezik egy előre megadott számmal!

A szükséges adatokat a program 3 parancssori paraméteren keresztül kapja.

Az első, egy egész értékű adat, mely a feladatban definiált számot jelzi, ezt kell valamilyen módon elérni a halmazelemek összegével.

A második, egy fájl neve - ez tartalmazza a kiinduló halmaz elemeit (inputfájl). A felépítése az alábbi: A fájl első sorában egy nemnegatív egész szám (N) áll - a kiinduló halmazunk elemszáma tehát N. A következő sorban összesen N db egész számot olvashatunk (pozitív és negatív egyaránt), melyek a halmaz elemeit jelölik (sorrendiséget nem kötünk meg köztük).

Egy megfelelő bemeneti fájl (például data.txt) ekkor:

6

3 34 4 17 5 2

A harmadik paraméter, annak a fájlnak a neve, melybe a feladat megoldása során kapott választ kell írni.

\section{Felhasználói dokumentáció}

A program futtatása a master futtatható fájl megnyitátásával történik. A program követel paramétereket, így egyszerűbb console paranccsal indítani.
Az első paraméter az elérni kívánt összeg, a második a bemeneti számokat tartalmazó fájl neve, illetve a harmadik a kimeneti fájl neve.

\subsection{Rendszer-követelmények, telepítés}

A programunk több platformon is futtatható, dinamikus függősége nincsen, bármelyik, manapság használt PC-n működik. Külön telepíteni nem szükséges, elég a futtatható állományt elhelyezni a számítógépen.

\subsection{A program használata}

A program használata egyszerű, de célszerű paranccsorból indítani, hiszen szükséges a futtatáshoz a paraméterek megadása. Két futtatható állomány (master és child) közül a master állományt kell futtatni, de mindkettő fájl jelenléte szükséges. Az első paraméter az elérni kívánt összeg, a második a bemeneti számokat tartalmazó fájl neve, illetve a harmadik a kimeneti fájl neve.


\section{Fejlesztői dokumentáció}

A program két programból áll. Egy master főprogramből, illetve egy child gyerekprogramból. PVM segítségével a master a megfelelő adatok megszerzése után kiadja a megoldás feladatát a child programnak. Ekkor a child az alapadattól függően eredményt küld vissza a master programnak, vagy további child programokat indít magából amíg az egyik ilyen child megoldást nem talál (ha létezik megoldás).

\subsection{Megoldási mód}

A program két részre lett osztva. A master program először beolvassa a fájlból a szükséges adatokat, majd az összes adatot átadja az 1. child programnak. A child program ekkor megnézi, hogy a felada készen van-e, amennyiven nincs, további 2 child programot indít magából, de azokat már önmagától kisebb feladatokkal látja el. Minden child program további programokat hoz létre egészen addig, ameddig a child program már az előtt az alapfeladat előtt áll, amiről egyértelműen eltudja dönteni, hogy megoldás-e, vagy sem. Miután ezeknek a számolásoknak vége, a bináris fát alkotott child programokon felfelé haladva megnézzük, hogy melyik ágon van megoldás. Amennyiben bármelyik ágon található megoldás, az 1. child program ezt visszaküldi a master programnak, amely a végén ezt kiírja a megfelelő fájlba.

\subsection{Implementáció}

A megoldási módban leírtakat 2 fájlba szervezve oldjuk meg. master.cpp felelős a master progarmért, child.cpp felelős a child programért. A két program közötti kommunikációt PVM biztosítja. A gyorsaság és biztonság érdekében a legtöbb változót a stack-en tárolunk, illetve C tömbök helyett std::vector<T>-t használunk. Mivel a PVM függvényei C típusokat várnak, az std::vector<T> adatai küldése előtt ebből egy lokális C tömböt hozunk létre. A két program közötti kommunikáció egyszerű és minimális. A master program elküldi az 1. child programnak az összeg értékét, a tömb hosszát, illetve a tömb értékeit. A child process ebből el tudja dönteni, hogy a feladat már megoldott-e, ha nem akkor további 2 child programot indít, aminek elküldi ugyan ezeket az adatokat, de már egy kisebb tömbre. Minden child process elküldi a szülő programjának, hogy talált-e megoldást. Ha a szülő program egy child, akkor 2 ilyen eredményt kap, ezek közül ha bármelyik megoldás, akkor az ő továbbküldött értéke is ezt fogja mutatni. Ahogy visszatérünk az 1. child programhoz, az már tudja, hogy található-e megoldás. Ezt az értéket elküldi a master programnak.

\subsection{Fordítás menete}

A fordításhoz szükséges egy Makefail.aimk, amely tartalmazza a szükséges paramétereket, illetve beállításokat. Ezzel fordítanunk kell a master.cpp, illetve child.cpp fájlt is.

\subsection{Tesztelés}

A program teszteléséhez több configuráción lett futtatva. Ehhez PVM-ben a blade-ek számát módosítottuk.

\begin{tabular}{ l r }
	Blade-ek száma & Másodperc \\ \hline
	1  &  0.0697199  \\
	2  &  0.0445748  \\
	3  &  0.0523889  \\
	4  &  0.0549406  \\
	5  &  0.0359066  \\
	6  &  0.0407191  \\
\end{tabular}


\end{document}